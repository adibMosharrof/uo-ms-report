\section{Prediction Module}

The prediction module in the Medifor Code base is analogous to a rib-eye steak, the prime  module of the system. This module loads image data,
 loads and trains a machine learning model, and makes predictions on test data. This document will outline how the prediction module works by
  initially starting at a high level and slowly delving into the intricate particulars.

The point of entry of the prediction module is the \emph{prediction\_runner.py} file. The module contains an abstract Prediction class, and three 
concrete classes — PixelPredictions, PatchPredictions and EnsemblePredictions. Similar to all the other module runners, the first step is
 to read in the configurations. The next step is to load a concrete Prediction class based on the configuration rules and call the 
 \emph{train\_predict} function. 

The \emph{train\_predict} function in the Predictions class directs the work being done in this module. It initializes the data generators, 
trains the model, produces the model output in the predict function, and finally calls the scoring module to get the scores. The 
child prediction classes implement methods for preparing the image references, initializing data generators, and reconstructing
 the image from the model output. The \emph{train\_model} function in the Predictions class has a call to the \emph{get\_architecture} function,
  which must also be implemented by the child prediction classes.

Two data generators, TrainDataGenerator and TestDataGenerator; are required by a Prediction class. As the name suggests, 
TrainDataGenerator is used to load data for training the model and TestDataGenerator is used to load the data for testing the model.
The data generators are located inside the \emph{data\_generators} folder. The project uses different data types: csv, pixel, patch and image.
There are test and train data generators for each of the data types. These data generators are fairly complex as they have a lot of parameters
in the  constructors. The key method in a data generator is the \emph{\_\_getitem\_\_} method, which returns two arrays. The first array is 
used as the input to the machine learning model and the second array is the ground truth. It is of paramount importance that the dimensions 
of the two arrays be consistent with what the machine learning model is expecting.

This module follows naming conventions when it comes to naming files and methods, which can be identified upon inspecting the files 
and methods. When creating new prediction or data generator classes, it is suggested to follow the pattern and place those files in 
the appropriate directories. If a new data type is introduced, a new Predictions class should be created and it should extend the 
abstract Prediction class and implement the abstract methods.
