\section{Related Work}


The initial work on this data focused on using traditional machine learning algorithms to perform image localization. In these models one 
row of data was considered to be a single pixel in an image. A data preprocessing step was applied to convert each image into a 1 dimensional
array, thus enabling the use of a single pixel as a row of data. The number of features in a model is equal to the number of indicators 
that have been selected to be used. As an example of a single row of data, we consider the top left pixel of an image, where we take the 
value of that pixel from each of the indicators and pass that into the model as input.

One of the major challenges in the previous work was to get a consistent scoring for the algorithms. As mentioned in the previous section, 
Mediscore was used to evaluate the performance of the models. Since it is a very time consuming step, a local scoring system was created which 
would provide a score for the models. However, there was no direct correlation between Mediscore and the local scoring method, which proved 
to be a major drawback when fine tuning models to increase performance. Tweaks in hyperparameters that would increase the score in the 
local scoring, would not produce the same behavior in from the results returned from Mediscore. A major step in our current work was to 
create a new local scoring system that would be fast and also be consistent to what Mediscore would produce, thus enabling us to perform 
experiments without having a tight coupling with Mediscore.

Previous work mainly focused on models with decision trees, regression and boosting. The scores produced from the different models are shown in 
figure \ref{fig:previous_results_table}

\begin{figure}
    \centering
    \csvautotabular{figures/previous_results_table.csv}
    \caption{Previous Results}
    \label{fig:previous_results_table}
\end{figure}

Get more ideas from Daniel about previous work

