
\section{Scoring}

As mentioned in the Data section, in the dataset, the proportion of manipulated pixels when compared to non-manipulated pixels is very small. Unblanced data generally causes algorithms 
to be biased towards the majority class, thus popular metrics like accuracy, which is the ratio between the number of correctly classified samples and the overall number of samples
(for example \cite{wang2007accurate}), will no longer be an accurate metric to use here. Accuracy would provide an overoptimistic estimation  of the classifier's ability on 
the majority class \cite{chicco2020advantages}. Consider a data set that has 10\% positive class and 90\% negative class. A naive classifier that always outputs the
majority class label will have a high accuracy of 0.90. 


The problem caused due to this imbalance can be addressed by using the Matthews correlation coefficient (MCC), a special case of the $\phi$ phi coefficient 
\cite{guilford1954psychometric}. MCC has been originally developed by B.W. Matthews for comparison of chemical structures \cite{matthews1975comparison}. However, it was 
re-proposed by Baldi and colleagues \cite{baldi2000assessing} in 2000 as a standard performance metric for machine learning with a natural extension to the multiclass
scenario \cite{gorodkin2004comparing}.

\subsection{Matthews Correlation Coefficient}

Matthews correlation coefficient is defined in terms of True Positive (TP), True Negative (TN), False Positive (FP) and False Negative (FN). These terms can be calculated 
from the confusion matrix as defined in Figure \ref{fig:confusion_matrix}. MCC is a contingency matrix method of calculating the \emph{Pearson product-moment correlation
coefficient} \cite{powers2020evaluation} and can be calculated using Equation \ref{eq:mcc}.

\begin{equation} \label{eq:mcc}
    MCC = \frac{TP \times TN - FP \times FN}{\sqrt{(TP+FP)(TP+FN)(TN+FP)(TN+FN)}}
\end{equation}

\begin{figure}
    \centering
    \def\arraystretch{1.5}% for the vertical padding
    \setlength{\tabcolsep}{1.5em} % for the horizontal padding
    \begin{tabular}{|c|c|c|c|}
        \hline
        & \multicolumn{3}{c|}{Prediction} \\
        \hline
        \parbox[t]{2mm}{\multirow{3}{*}{\rotatebox[origin=c]{90}{Actual}}} & &P &N \\
        \cline{2-4}
        &  P & TP& FN \\
        \cline{2-4}
        & N& FP& TN \\
        \hline
        \end{tabular}

    \caption{The Standard Confusion Matrix }
    \medskip
    \footnotesize  Positives (P) and Negatives (N). True Positives(TP) and True Negatives (TN) are the correct predictions, while False Negatives (FN) and False Positive (FP)
    are the incorrect predictions
    
    
    \label{fig:confusion_matrix}
\end{figure}

\subsection{Mediscore}

The scoring mechanism in the Mediscore project has a concept of no score region that was not considered in the previous work. The idea behind the no score region is to 
ignore the regions on the boundary of the manipulated and non-manipulated regions. If an algorithm can identify the manipulated region in the image, it could still 
mislabel a lot of pixels in the boundary region, which could incurr a high penalty to the algorithm and give it a score that does not accurately reflect its performance.
To get a more fair score, the score is calculated by considering the pixels in the image excluding the pixels that are present in the no score region. In order to produce
the no score region, two of the core morphologic operations; erosion and dilasion, are applied to the image. A difference operation is performed on the images produced 
from erosion and dilation, which is followed by a binary inverse thresholding to produce the set of pixels that constitute the no score region.

\subsection{Dilation}

Applying dilation $\oplus$ to an image expands the shape of the image. Let $X$ be a gray scale shape and $B$ be a symmetric structuring element. The dilation operation 
on these two elements, $X \oplus B$ can be defined as 

\begin{equation} \label{eq:dilation}
    X \oplus B = X + b = \{ x+ b : x \in X \& b \in B \} 
\end{equation}


The output of dilation is the set of translated points such that translate of the reflected structuring element has a non-empty intersection with $X$. Equation 
\ref{eq:dilation} is based on obtaining the refelction of $B$ about its origin and shifting this reflection by $b$. The dilation of $X$ by $B$ is the set of all the 
displacements, $b$, such that $x$ and $b$ overlap by atleast one element \cite{tambe2013image}.

\subsection{Erosion}

Erosion $\Theta$ is generally applied to images for eliminating irrelevent details in terms of size. 

\begin{equation} \label{eq:erosion}
    X \Theta B = X - b = \{ z: (  B + z )  \in X \} 
\end{equation}

The output of erosion is the set of translation points such that the translated structuring element is contained in the input set $X$. Equation \ref{eq:erosion} indicates 
that the erosion of $X$ by $B$ is the set of all points $b$ such that $B$, translated by $b$, contain $X$ \cite{tambe2013image}. The general output of erosion is that it 
shrinks the shape of the image. 

\subsection{No Score Region}

The boundary no score region is produce by performing two operations sequentially, which are defined in Equation \ref{eq:morphed} and \ref{eq:bin_inv}
Equation \ref{eq:bin_inv} is basically a binaray inverse thresholding.

\begin{equation} \label{eq:morphed}
    \psi = ( X \Theta B ) - ( X \oplus B)
\end{equation}

\begin{equation} \label{eq:bin_inv}
    \text{NoScore}(x,y)=\begin{cases}
        0, & \text{if } \psi(x,y) > 0  \\
        1, & \text{otherwise}
     \end{cases}
\end{equation}

\subsection{Thresholded MCC}

The output of the algorithms is a grayscale image where pixel values are in the range $[0-255]$ and the ground truth is an image which has pixel values of $0$ or $1$, which 
represent manipulated and non-manipulated pixels respectively. The output image is converted from a 2 dimensional array into a 1 dimensional array and the pixel indices that
are in the no score region are filtered out to produce a list of pixels that are to be used in scoring. The indices in this list are called the scoring indices which will 
be used to calculate the MCC score.

The pixel values in the scoring indices list need to be converted to have a value of either $0$ or $1$. This conversion is done using a binary thresholding on the scoring 
indices list, where the threshold values are in the range $[0-255]$. Thresholding is applied using every value in the range and the output produced is used with the ground truth
to produce an MCC score for that particular threholded value. The maximum MCC value obtained is selected as the MCC score for the image.