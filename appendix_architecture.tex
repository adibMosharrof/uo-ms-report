\section{Architecture Module}

The architectures module contains the implementations of the different machine learning models used. 
Each class in this module defines a machine learning model and must implement a \emph{get\_model} method, which returns a model object. 
This module cannot be run independently; it exists to serve the predictions module. Based on the model name provided configuration
 of the predictions module, a model class will be instantiated and the \emph{get\_model} method called during  the training phase.

A model class serves only one data format, therefore when creating or investigating an existing model, the data format must also
 be explored. Neural network models have been implemented for pixel and image data formats. Logistic regression models have been
  implemented for the pixel data formats and the UNet model has been implemented for the patched image data format.

The Keras library has been used to create UNet and Neural Network models, whereas SkLearn library has been used to create Logistic
 regression models. The model objects returned from these two libraries are not consistent, thus the prediction module has to 
 account for the differences. Keras provides the ability to perform training on small batches, whereas Sklearn expects the whole
  data to be loaded at once.

The UNet and Neural Network models have been parameterized to be customizable. The full list of parameters and details about 
specific parts of the model can be found in the respective classes. The logistic regression model just returns a SkLearn model
 object without any such customizations.
